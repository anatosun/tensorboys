\documentclass{homework}
\usepackage[utf8]{inputenc}
\usepackage{enumitem}
\usepackage{amsmath}
\usepackage{amssymb}
\usepackage{mathtools}
\usepackage{braket}
\usepackage{listings}   
\usepackage{hyperref}                      % to allow urls and inner references

\newcommand{\hwtype}{Lessons Learned}
\newcommand{\hwclass}{Pattern Recognition}
\newcommand{\hwlecture}[0]{}
\newcommand{\hwsection}[0]{}
\newcommand{\quotes}[1]{``#1"}
\newcommand{\hwname}{Antoine Demont \\
Boris Mottet \\
François-Xavier Wicht \\
Martin Poplawski \\
Vincent Carrel}
\newcommand{\hwemail}{\href{mailto:tensorboys@unifr.ch}{\color{black}tensorboys@unifr.ch}}
% CHANGE THESE ONLY ONCE PER SERIES
\newcommand{\hwnum}{}
\hypersetup{
  colorlinks=true,
  linkcolor=black,
  citecolor=green,
  filecolor=magenta,
  urlcolor=blue,
  pdftitle={\hwclass \hwtype \hwnum},
  pdfpagemode=FullScreen,
}
\begin{document}

\maketitle



% \question*{Virtualisation}
Through this project we were first able to learn to coordinate among a relatively large group without necessarily having direct contact. The challenge here was to distribute the tasks according to each other's strengths, weaknesses and interests. It was sometimes difficult to establish an equitable allocation of tasks. We therefore decided to structure this report as a personal testimony for each member of the group.

\question*{Antoine Demont}
\question*{Boris Mottet}
\question*{François-Xavier Wicht}
\question*{Martin Poplawski}
\question*{Vincent Carrel}
\end{document}
