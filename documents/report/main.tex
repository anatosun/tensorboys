\documentclass{homework}
\usepackage[utf8]{inputenc}
\usepackage{enumitem}
\usepackage{amsmath}
\usepackage{amssymb}
\usepackage{mathtools}
\usepackage{braket}
\usepackage{listings}   
\usepackage{hyperref}                      % to allow urls and inner references

\newcommand{\hwtype}{Lessons Learned}
\newcommand{\hwclass}{Pattern Recognition}
\newcommand{\hwlecture}[0]{}
\newcommand{\hwsection}[0]{}
\newcommand{\quotes}[1]{``#1"}
\newcommand{\hwname}{Antoine Demont \\
Boris Mottet \\
François-Xavier Wicht \\
Martin Poplawski \\
Vincent Carrel}
\newcommand{\hwemail}{\href{mailto:tensorboys@unifr.ch}{\color{black}tensorboys@unifr.ch}}
% CHANGE THESE ONLY ONCE PER SERIES
\newcommand{\hwnum}{}
\hypersetup{
  colorlinks=true,
  linkcolor=black,
  citecolor=green,
  filecolor=magenta,
  urlcolor=blue,
  pdftitle={\hwclass \hwtype \hwnum},
  pdfpagemode=FullScreen,
}
\begin{document}

\maketitle



% \question*{Virtualisation}
Through this project we were first able to learn to coordinate among a relatively large group without necessarily having direct contact. The challenge here was to distribute the tasks according to each other's strengths, weaknesses and interests. It was sometimes difficult to establish an equitable allocation of tasks. We therefore decided to structure this report as a personal testimony for each member of the group. Each statement is here addressed in the first person.

\question*{Antoine Demont}
\question*{Boris Mottet}
\question*{François-Xavier Wicht}
These projects were interesting in that they allowed us to touch on many subjects and to work with different libraries and methods. I noticed that some seemingly difficult classification problems, such as detecting whether a signature has been forged, could easily be solved using techniques learned in class. Moreover, these different projects allowed me to realise some typical trade-offs like between precision and recall. Although now obvious, it is clear that classification can never be perfect, either we manage to capture all the elements by decreasing the precision of our classifier or we omit some elements but increase our precision. This kind of classification problem can be applied to many other problems.

I was able to work in another course on the de-anonymisation of users in the Bitcoin P2P network. There, for the adversary, it is also a problem of classification. She has to map each transaction to a server. There is indeed deanonymisation when a transaction can be mapped to a server in the network. So it's a bit of a classification problem with one item per class at each time. The techniques described in the paper were all trivial for me thanks to this project and I could even implement some of them easily. Classification problems are everywhere and this project helped me to build a toolbox for my future professional and academic life.

\question*{Martin Poplawski}
\question*{Vincent Carrel}
\end{document}
